\documentclass[twocolumn,a4paper,11pt]{article}
\usepackage[]{times}
\usepackage[utf8]{inputenc}
\usepackage[IL2]{fontenc}
\usepackage[czech]{babel}
\usepackage[]{alltt}
\usepackage[hidelinks]{hyperref}
\usepackage[]{dirtytalk}
\usepackage[usenames,dvipsnames]{color}
\usepackage[left=1.4cm,text={18.2cm, 25.2cm},top=2.3cm]{geometry}
\usepackage[]{amsmath}

\begin{document}

\begin{center}
    \Huge
    \textsc{Vysoké učení technické v Brně \\ Fakulta informačních technologií}
    \vspace{\stretch{0.382}}
    \textsc{Typografie a publikování -- 2. projekt \\ Sazba dokumentů a matematických výrazů}
    \vspace{\stretch{0.618}}
\end{center}

{\date{2023} \hfill \author{Roland Schulz (xschul06)} }
\thispagestyle{empty}
\newpage

\section{Úvod}
\label{sec:intro}
V této úloze si vyzkoušíme sazbu titulní strany, matematických vzorců, prostředí a dalších textových struktur obvyklých pro technicky zaměřené texty -- například Definice~\ref{subsec: definice} nebo rovnice~\eqref{eq:integrals} na straně~\pageref{sec: rovnice}. Pro vytvoření těchto odkazů používáme kombinace příkazů \verb|\label|, \verb|\ref|, \verb|\eqref| a \verb|\pageref|. Před odkazy patří nezlomitelná mezera. Pro zvýrazňování textu jsou zde několikrát použity příkazy \verb|\verb| a \verb|\emph|.

Na titulní straně je použito prostředí \verb|titlepage| a sázení nadpisu podle optického středu s využitím \emph{přesného} zlatého řezu. Tento postup byl probírán na přednášce. Dále jsou na titulní straně použity čtyři různé velikosti písma a mezi dvojicemi řádků textu je použito odřádkování se zadanou relativní velikostí 0,5 em a 0,4 em (Nezapomeňte použít správný typ mezery mezi číslem a jednotkou.).

\section{Matematický text}
\label{sec:mattext}
V této sekci se podíváme na sázení matematických symbolů a výrazů v plynulém textu pomocí prostředí \verb|math|.
Definice a věty sázíme pomocí příkazu \verb|\newtheorem| s využitím balíku \verb|amsthm|.
Někdy je vhodné použít konstrukci \verb|${}$| nebo \verb|\mbox{}|, která říká, že (matematický) text nemá být zalomen.
\newtheorem{ZA}{Zásobníkový automat} (ZA) je definován jako sedmice tvaru, $A = (Q, \Sigma, \Gamma, \delta, q_0 , Z_0 , F)$, kde: \begin{itemize} \item $Q$ je konečná množina vnitřních (řídicích) stavů, \item $\Sigma$ je konečná vstupní abeceda, \item $\Gamma$ je konečná zásobníková abeceda, \item $\delta$ je přechodová funkce, \item $q_0 \in Q$ je počáteční stav, $Z_0 \in \Gamma$ je startovací symbol zásobníku a $F \subseteq Q$ je množina koncových stavů.\end{itemize}
Nechť $P = (Q,\Sigma,\Gamma,\delta,q_0,Z_0,F)$ je ZA. Konfigurací nazveme trojici $(q,w,\alpha) \in Q \times \Sigma^*\times \Gamma^*$, kde $q$ je aktuální stav vnitřního řízení, $w$ je dosud nezpracovaná část vstupního řetězce a $\alpha = Z_{i_1}Z_{i_2}\dots Z_{i_k}$ je obsah zásobníku.

Podsekce obsahující definici a větu
Řetězec nad abecedou je přijat ZA jestliže pro nějaké a. Množina A je jazyk přijímaný ZA. Třída jazyků, které jsou přijímány ZA, odpovídá bezkontextovým jazykům.

\section{Rovnice}
\label{sec:rovnice}
Složitější matematické formulace sázíme mimo plynulý text pomocí prostředí \verb|displaymath|. Lze umístit i několik výrazů na jeden řádek, ale pak je třeba tyto vhodně oddělit, například příkazem \verb|\quad|.
\[1^{2^3}\neq \Delta^1_{\Delta^2_{\Delta^3}}\quad y^{11}_{22}-\sqrt[9]{x+\sqrt[7]{y}}\quad x < y_1\leq y^2\]
V rovnici jsou využity tři typy závorek s různou \emph{explicitně} definovanou velikostí. Také nepřehlédněte, že nasledující tři rovnice mají zarovnaná rovnítka, a použijte k tomuto účelu vhodné prostředí.
\begin{eqnarray}
    -\cos^2\beta & = & \frac{\frac{\frac{1}{x}+\frac{1}{3}}{y}+1000}{\prod\limits_{j=2}^8 q_j} \\
    \left(\left\{b\star[3 \div 4]\circ a\right\}^\frac{2}{3}\right) & = & \log_10 x \\
    \int_a^b f(x)\,\mathrm{d}x & = & \int_c^d f(y)\,\mathrm{d}y
\end{eqnarray}
V této větě vidíme, jak vypadá implicitní vysázení limity $\lim_{m\rightarrow \infty} f(m)$ v normálním odstavci textu. Podobně je to i s dalšími symboly jako $\bigcup_{N\in\mathcal{M}} N$ či $\sum^{m}_{i=1} x^2_i$.
S vynucením méně úsporné sazby příkazem \verb|\limits| budou vzorce vysázeny v podobě $\lim\limits_{m\rightarrow \infty} f(m)$ a $\sum\limits_{i=1}^{m} x^{4}_{i}$.

\section{Matice}
\label{sec:matice}
Pro sázení matic se velmi často používá prostředí \verb|array| a závorky (\verb|\left|, \verb|\right|).
\begin{eqnarray*}
\mathbf{B} = \left|
\begin{array}{cccc}
    b_{11} & b_{12} & \cdots & b_{1n} \\
    b_{21} & b_{22} & \cdots & b_{2n} \\
    \vdots & \vdots & \ddots & \vdots \\
    b_{m1} & b_{m2} & \cdots & b_{mn} \\
\end{array}
\right|
= \left|\begin{array}{cc} t & u \\ v & w \\ \right|\end{array} = tw-uv \\
\mathbb{X} = \mathbf{Y} \Longleftrightarrow \left[\begin{array}{ccc}
    & \Omega + \Delta & \hat{\psi} \\
    \overrightarrow{\pi} & \omega & \\
\end{array}\right] \neq 42
\end{eqnarray*}
Prostředí \verb|array| lze úspěšně využít i jinde, například na pravé straně následující rovnice.
Kombinační číslo na levé straně vysázejte pomocí příkazu \verb|\binom|.
\[
\binom{n}{k} = \left\{ \begin{array}{cc}
    0 & \mathrm{pro} k<0 \\
    \frac{n!}{k!(n-k)!} & \mathrm{pro} 0\leq k \leq n \\
    0 & \mathrm{pro} k>0 \\
\end{array}
\]

\end{document}