\documentclass[a4paper,11pt]{article}
\usepackage[]{times}
\usepackage[utf8]{inputenc}
\usepackage[IL2]{fontenc}
\usepackage[czech]{babel}
\usepackage[]{alltt}
\usepackage[hidelinks]{hyperref}
\usepackage[]{dirtytalk}
\usepackage[usenames,dvipsnames]{color}
\usepackage[left=2cm,text={17cm, 24cm},top=3cm]{geometry}
\usepackage{amsmath}
\usepackage{amsfonts}
\usepackage{amsthm}
\usepackage[czech,longend,linesnumbered,algoruled,noline]{algorithm2e}
\usepackage{multirow}
\usepackage{multicol}
\usepackage{graphicx}

%FOR_DIFF_PURPOSES
%\usepackage[]{xcolor}
%\color{red}

\newcommand*{\logictable}[7]{
    \begin{tabular}{|c|c|c|c|c|c|}
        \hline
        \multicolumn{2}{|c|}{\multirow{2}{*}{$#1 #2 #3$}} & \multicolumn{4}{|c|}{$#3$} \\\cline{3-6}
        \multicolumn{2}{|c|}{}                             & \textbf{P} & \textbf{O} & \textbf{X} & \textbf{N} \\\hline
        \multirow{4}{*}{$#1$}& \textbf{P}                  & #4 \\
                             & \textbf{O}                  & #5 \\
                             & \textbf{X}                  & #6 \\
                             & \textbf{N}                  & #7 \\\hline
    \end{tabular}
}

\begin{document}
\begin{titlepage}
    \begin{center}
        {\Huge \textsc{Vysoké učení technické v~Brně} \\[0.5em]} {\huge \textsc{Fakulta informačních technologií}} \\
        \vspace{\stretch{0.382}}
        {\LARGE Typografie a publikování\,--\,3. projekt \\[0.4em] Tabulky a obrázky }\\
        \vspace{\stretch{0.612}}
    \end{center}

    {\Large \today \hfill Roland Schulz (xschul06)}
    \thispagestyle{empty}
\end{titlepage}

\pagenumbering{arabic}

\section{Úvodní strana}
Název práce umístěte do zlatého řezu a nezapoměňte uvést \uv{dnešní} (today) datum a vaše jméno a přijmení.

\section{Tabulky}
Pro sázení tabulek můžeme použít buď prostředí \verb|tabbing| nebo prostředí \verb|tabular|.
\subsection{Prostředí \texttt{tabbing}}
Při použití \verb|tabbing| vypadá tabulka následovně:

\begin{table}[ht]
    \begin{tabbing}
        Ovoce \= Cena \= Množství \\
        Jablka \= 25,90 \= 3\,kg \\
        Hrušky \= 27,40 \= 2,5\,kg \\ 
        Vodní melouny \= 35,-- \= 1\,kus \kill
    \end{tabbing}
\end{table}
Toto prostředí se dá také použít pro sázení algoritmů, ovšem vhodnější je použít prostředí \verb|algorithm| nebo \verb|algorithm2e| (viz sekce \ref{sec:algoritmy}).

\subsection{Prostředí \texttt{tabular}}
Další možností, jak vytvořit tabulku, je použít prostředí \verb|tabular|. Tabulky pak budou vypadat takto\footnote{Kdyby byl problem s \texttt{cline}, zkuste se podivat třeba sem: http://www.abclinuxu.cz/tex/poradna/show/325037.}
\begin{table}[ht]
    \centering
    \catcode`\-=12
    \begin{tabular}{|c|c|c|}
        \hline
        \multirow{2}{*}{Měna} & \multicolumn{2}{c|}{Cena} \\\cline{2-3}
                              & nákup & prodej  \\\hline
                          EUR & 22,705 & 25,242 \\
                          GBP & 25,931 & 28,828 \\
                          USD & 21,347 & 23,732 \\\hline
    \end{tabular}
    \caption{Tabulka kurzů k dnešnímu dni}
    \label{tab:kurzy}
\end{table}

\begin{table}[ht]
    \catcode`\-=12
    \centering
    \begin{tabular}{|c|c|}
        \hline
        $A$ & $\neg A$ \\\hline
        \textbf{P} & N \\\hline
        \textbf{O} & O \\\hline
        \textbf{X} & X \\\hline
        \textbf{N} & P \\\hline
    \end{tabular}
    \smallskip
    \logictable{A}{\land}{B}
        {P & O & X & N}
        {O & O & N & N}
        {X & N & X & N}
        {N & N & N & N}
    \smallskip
    \logictable{A}{\lor}{B}
        {P & P & P & P}
        {P & O & P & O}
        {P & P & X & X}
        {P & O & X & N}
    \smallskip
    \logictable{A}{\rightarrow}{B}
        {P & O & X & N}
        {P & O & P & O}
        {P & P & X & X}
        {P & P & P & P}

    \caption{Protože Kleeneho trojhodnotová logika už je \uv{zastralá}, uvádíme si zde příklad čtyřhodnotové logiky}
\end{table}

\section{Algoritmy}
\label{sec:algoritmy}

Pokud budeme chtít vysázet algoritmus, můžeme použít prostředí \verb|algorithm|\footnote{} nebo \verb|algorithm2e|\footnote{}.
Příklad použítí prostředí \verb|algorithm2e| viz \ref{alg:fastslam}

\begin{algorithm}
    \caption{F{\footnotesize AST}SLAM}
    \label{alg:fastslam}
    \KwIn{$(X_{t-1}, u_t, z_t)$}
    \KwOut{$X_t$}
    \vspace{0.5em}
    $\overline{X_t} = X_t = 0$\;
    \For{$k = 1$ to $M$}{
        $x^{[k]}_t = \emph{sample\_motion\_model}(u_t, x^{[k]}_{t-1})$\;
        $\omega^{[k]}_t = \emph{measurement\_model}(z_t, x^{[k]}_t, m_{t-1})$\;
        $m^{[k]}_t = updated\_occupancy\_grid(z_t, x^{[k]}_t, m^{[k]_{t-1}})$\;
        $\overline{X_t} = \overline{X_t} + \langle x^{[m]}_x, \omega^{[m]}_t \rangle$\;
    }

    \For{$k = 1$ to $M$}{
        draw $i$ with probability $\approx \omega^{[i]}_t$\;
        add $\langle x^{[k]_x, m^{[k]}_t}\rangle$ to $X_t$\;
    }
    \KwRet{$X_t$}

\end{algorithm}

\section{Obrázky}
Do našich článků můžeme samozřejmě vkládat obrázky. Pokud je obrázkem fotografie, můžeme klidně použít bitmapový soubor.
Pokud by to ale mělo být nějaké schéma nebo něco podobného, je dobrým zvykem takovýto obrázek vytvořit vektorově.

\begin{figure}[!ht]
    \centering
    \includegraphics{etiopan.eps}
    \reflectbox{\includegraphics{etiopan.eps}}
    \caption{Malý Etiopánek a jeho bratříček}
\end{figure}


\end{document}