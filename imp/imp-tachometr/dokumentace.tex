\documentclass[twocolumn,a4paper,10pt]{article}
\usepackage[utf8]{inputenc}
\usepackage[T1]{fontenc}
\usepackage[czech]{babel}
\usepackage[]{alltt}
\usepackage{float}
\usepackage[hidelinks]{hyperref}
\usepackage[inkscapeformat=png]{svg}
\usepackage[]{dirtytalk}
\usepackage{listings}
\usepackage{minted}
\usepackage[left=1.6cm,text={18cm, 24.7cm},top=1.5cm]{geometry}
\usepackage{tikz}
\usetikzlibrary{automata, positioning, arrows}
\usepackage{graphicx}
\usepackage{multirow}

\newcommand*{\email}[1]{%
    \href{mailto:#1}{#1}
}

\begin{document}

\twocolumn[
		\begin{@twocolumnfalse}
			\begin{center}
				{ \includegraphics[width=0.5\linewidth]{FIT_barevne_RGB_CZ.png} } \\
    
				{\Large
					Vysoké učení technické v~Brně \\
					Fakulta informačních technologií \\
				}

				{\LARGE
					Mikroprocesorové a~vestavěné systémy \\
					Projekt\,--\,Tachometr \\[0.4cm]
				}

				{\large
					Roland Schulz \\
                    \email{xschul06@stud.fit.vutbr.cz} \\ 
					\today
				}
			\end{center}
		\end{@twocolumnfalse}
	]


\section{Představení}\label{sec:intro}
Projekt se zabývá sestrojením cyklopočítače zobrazujícího aktuální rychlost, průměrnou rychlost a ujetou vzdálenost na displeji.

Menu na displeji cyklopočítače lze ovládat jedním z dostupných tlačítek, druhé tlačítko simuluje otočení kola.

Cyklopočítač umožňuje přístup skrze webové rozhraní poskytující vizualizací aktuálních dat, konfiguraci cyklopočítače nebo přístup k strojově čitelným datům ve formátu JSON.

Cyklopočítač lze použít samostatně nebo v kombinaci s chytrým zařízením pro centralizovaný sběr dat z tachometru.

\section{Schéma zapojení}\label{sec:schema}
Displej je zapojen podle doporučení ze zadání.

Zapojení všech periferií k mikrokontroleru je zdokumentováno graficky na obrázku \ref{fig:schematics} nebo v tabulce \ref{table:schematics}.

\begin{center}
    \begin{figure}[h]
        \centering
        \includegraphics[width=0.7\linewidth,keepaspectratio]{schematics.pdf}
        \caption{Schéma zapojení graficky}
        \label{fig:schematics}
    \end{figure}
\end{center}
\subsection{Zapojení pinů}\label{sec:pins}
% Please add the following required packages to your document preamble:
% \usepackage{multirow}

\begin{center}
    \begin{table}[h]
        \begin{tabular}{|c|l|l|l|}
            \hline
            Komponenta                            & Vývod            & MCU Pin\\ \hline
            \multirow{5}{*}{SSD1306 SPI displej}  & MOSI             & 23  \\ 
                                                  & CLK              & 18  \\ 
                                                  & DC               & 27  \\ 
                                                  & CS               & 5   \\ 
                                                  & RESET            & 17  \\ \hline
            \multicolumn{1}{|l|}{Tlačítko Rotace} & -                & 26  (Input pullup) \\ \hline
            \multicolumn{1}{|l|}{Tlačítko Menu}   & -                & 25  (Input pullup) \\ \hline
        \end{tabular}
        \caption{Schéma zapojení vyjádřeno tabulkou}
        \label{table:schematics}
    \end{table}
\end{center}

\section{Použití}
\begin{center}
    \begin{figure}[h]
        \centering
        \includegraphics[width=0.7\linewidth,keepaspectratio]{menu_state_machine.pdf}
        \caption{Stavový automat přechodů mezi zobrazením v menu}
        \label{fig:menu_FSM}
    \end{figure}
\end{center}
Menu cyklopočítače se skládá ze seřazených záložek, mezi kterými může uživatel krátkým stisknutím tlačítka menu přepínat. Obnovení dat zobrazených na aktuální záložce probíhá každou vteřinu.

Většina záložek zpřístupňuje i alternativní zobrazení nebo funkci skrze dlouhý stisk tlačítka menu (stisk trvající alespoň 2 sekundy).

Přepínání mezi záložkami je implementováno pomocí stavového automatu reprezentovaného na obrázku \ref{fig:menu_FSM}, kde přechody představují krátký stisk tlačítka menu a stavy představují jednotlivé záložky v menu s příkladnou ukázkou dané záložky.

\subsection{Záložky menu}
Přehled záložek dostupných v nabídce menu cyklopočítače.

Zobrazené údaje na záložkách se v závislosti na zobrazovaných datech snaží přizpůsobit dostupnému místu na displeji pro zajištění čitelnosti (např. úpravou jednotek nebo posunutím desetinné čárky).

\begin{itemize}
    \item \textbf{Aktuální rychlost a průměrná rychlost} \\
    Záložka ve výchozím stavu po zapnutí zobrazuje aktuální i průměrnou rychlost kola v kilometrech za hodinu.
    
    Dlouhým stiskem lze přepínat mezi zobrazením rychlosti v kilometrech za hodinu nebo v mílích za hodinu.

    Pro výpočet průměrné rychlosti se použije posledních 10 údajů o aktuální rychlosti uchovávaných v cyklickém bufferu ve volatilní paměti.

    \item \textbf{Ujetá vzdálenost} \\
    Záložka zobrazuje celkovou ujetou vzdálenost v metrech od posledního smazání. Jednotka ujeté vzdálenosti se automaticky nastaví na kilometry v případě že ujetá vzdálenost přesáhne 900 metrů.
    
    Dlouhým stiskem lze ujetou vzdálenost nastavit na hodnotu 0 (resetovat).

    Ujetá vzdálenost se při každé změně ukládá do perzistentní paměti, ze které je při opakovaném zapnutí cyklopočítače dostupná.

    \item \textbf{Otáčkoměr} \\
    Záložka zobrazuje aktuální otáčky kola za minutu.

    \item \textbf{Funkce sdílení dat} \\
    Záložka umožňuje přístup ke sdílení dat z tachometru pomocí webového rozhraní.
    
    Funkce poskytované touto záložkou je implementovaná pomocí stavového automatu na obrázku \ref{fig:AP_FSM} (předpokládá korektní připojení a přechody mezi stavy bez nutnosti předčasného ukončení sdílení), kdy sdílení dat je ve výchozím stavu vypnuté a přechod mezi stavy se provádí pomocí dlouhého stisknutí tlačítka menu.

    Záložka se trvale nachází pouze ve stavu vypnuto, sdílení pomocí přístupového bodu WiFi cyklopočítače (přístupové údaje k připojení na přístupový bod, IP adresa cyklopočítače a počet připojených zařízení k přístupovému bodu jsou v tomto stavu zobrazené na záložce), sdílení pomocí připojení na dostupnou WiFi síť podle předešle nakonfigurovaných přístupových údajů a zasílání dat na "cloud" metodu přibližně každou vteřinu (pokud se cyklopočítači nepovede připojit na takto nakonfigurovanou WiFi síť po 120 vteřinách, sdílení se automaticky přepne do stavu vypnuto).

    Stav sdílení dat je nezávislý od aktuálně zobrazené záložky, lze tedy po zapnutí sdílení dat, kterýmkoliv ze způsobů, přejít na libovolnou další záložku a sdílení bude probíhat nadále.

    Sdílení dat a konfigurace přístupových údajů je dále popsáno v sekci \ref{subsec:web}.
\end{itemize}

\subsection{Webové rozhraní}\label{subsec:web}
Cyklopočítač poskytuje jednoduché webové rozhraní pro zobrazení aktuálních hodnot ujeté vzdálenosti a rychlosti, nastavení sdílení zasíláním dat na "cloud" metodu, přístup k strojově čitelným datům z cyklopočítače a smazání ujeté vzdálenosti.

Pro přístup je třeba mít zařízení připojené na stejnou WiFi síť jako cyklopočítač (případně využít přístupového bodu cyklopočítače).

Adresa pro přístup k webovému rozhraní se zobrazí na displeji.

Reset vzdálenosti lze provést nezávisle na aktuální záložce menu cyklopočítače.

Strojovš čitelná data ve formátu JSON odpovídají formátu zobrazeném v příkladném soupisu \ref{code:json_data}.

\begin{center}
    \begin{figure}[H]
        \centering
        \includegraphics[width=0.7\linewidth,keepaspectratio]{admin.png}
        \caption{Hlavní menu webového rozhraní s ovládacími prvky}
        \label{fig:admin}
    \end{figure}
\end{center}

\begin{center}
    \begin{figure}[H]
        \centering
        \includegraphics[width=0.7\linewidth,keepaspectratio]{station_setup.png}
        \caption{Formulář webového rozhraní pro správu zasílání dat na cloud}
        \label{fig:cloud}
    \end{figure}
\end{center}

\begin{center}
    \begin{figure}[H]
        \centering
        \includegraphics[width=0.7\linewidth,keepaspectratio]{AP_state_machine.pdf}
        \caption{Stavový automat sdílení dat skrze webové rozhraní}
        \label{fig:AP_FSM}
    \end{figure}
\end{center}

\section{Zajímavosti implementace}\label{sec:impl_interests}
Výpočet aktuální rychlosti probíhá na základě poslední prodlevy mezi rotacemi kola. Prodleva mezi posledními dvěma rotacemi však není postačující v případě kdy kolo začne zpomalovat nebo úplně zastaví.

Pokud by se implementace řídila pouze poslední prodlevou mezi rotacemi, zobrazovaná aktuální rychlost má tendenci náhlých a výrazných skoků při zpomalování. V případě úplného zastavení kola by také cyklopočítač ukazoval neustále stejnou rychlost neodpovídající reálnému stavu zastaveného kola.

Implementace se taky nemůže řídit pouze prodlevou mezi poslední rotací a současností, v takovém případě by rotace krátce následovaná obnovením dat na displeji měla za následek výrazný skok zrychlení neodpovídající realitě.

Výpočet aktuální rychlosti se tak děje při každém obnovení dat záložek na displeji (každou vteřinu) a rozšiřuje se o prodlevu mezi poslední rotací kola a současností (čas poslední obnovy dat).

Pro výpočet aktuální rychlosti se použije \textbf{delší} z prodlev.

Zpomalují-li otáčky kola, vybere se pro výpočet zpomalené rychlosti prodleva od poslední rotace kola po současnost.

Zrychlují-li otáčky kola, použije se pro výpočet aktuální rychlosti prodleva mezi posledními dvěma rotacemi.

V případě shodných prodlev nezáleží na vybrané prodlevě pro správný výpočet.

Postup výběru prodlevy pro správný výpočet a zobrazení aktuální rychlosti je popsán graficky na obrázku \ref{fig:t_diff}.

\begin{center}
    \begin{figure}[h]
        \centering
        \includegraphics[width=1\linewidth,keepaspectratio]{delta_time_select.pdf}
        \caption{Výběr $\Delta t$ pro správné vykreslení zpomalování}
        \label{fig:t_diff}
    \end{figure}
\end{center}

\section{Použité technologie}\label{sec:technologies}
Programové i technické vybavení použité při vývoji projektu a programové i technické závislosti potřebné pro korektní funkčnost projektu.

\subsection{Hardware}\label{sec:dependencies_hw}
Projekt byl vyvíjen a testován na desce WeMos D1 R32 UNO ESP32 s připojeným displejem SSD1306 přes rozhraní SPI.

Dále je využito dvou mikrospínačů, z nichž jeden simuluje rotaci kola připojeného k tachometru a druhý slouží k ovládání tachometru.

Schéma zapojení hardwarových prostředků je popsáno graficky na obrázku \ref{fig:schematics} nebo v tabulce \ref{table:schematics}.

\subsection{Software}\label{sec:dependencies_sw}
Pro vývoj projektu bylo použito vývojové platformy Arduino ve vývojovém prostředí \footnote{https://platformio.org}{PlatformIO}.

Významné knihovny použité v projektu:
\begin{itemize}
    \item \texttt{Adafruit\_SSD1306.h} \\komunikace s OLED displejem
    \item \texttt{Preferences.h} \\perzistentní paměť
    \item \texttt{WebServer.h} \\webové rozhraní
    \item \texttt{WiFi.h} \\použití WiFi antény pro potřeby webového rozhraní a "cloud" metody
    \item \texttt{HTTPClient.h} \\zasílání dat tachometru na "cloud" metodu
\end{itemize}

\section{Přednosti a omezení}\label{sec:limits_benefits}
\subsection{Přednosti}\label{subsec:benefits}
Tachometr je od začátku koncipován pro využití jako jednoduchý cyklistický počítač pro použití v kombinaci s přenosným WiFi přístupovým bodem pro centralizovaný sběr dat.

Uživatelské rozhraní obsahuje elementy dostatečné velikosti pro komfortní čitelnost i za jízdy na jízdním kole.

Systém záložek poskytuje uživateli přímočarý způsob používání tachometru i za jízdy.

Záložky menu na displeji cyklopočítače podporují alternativní funkce při dlouhém stisku po dobu dvou sekund.

Jednotku rychlosti pro zobrazení aktuální a průměrné rychlosti lze uživatelsky konfigurovat dlouhým stisknutím v záložce zobrazující rychlosti. Dlouhým stisknutím lze přepínat mezi jednotkou míle za hodinu (Mph) a kilometry za hodinu (Km/h).

Tachometr podporuje připojení až 4 zařízení v režimu \footnote{Access Point - přístupový bod WiFi}{AP} pro konfiguraci tachometru nebo čtení dat pomocí dotazování.

V režimu stanice je tachometr připojen k nakonfigurované a dostupné síti WiFi, v tomto režimu zároveň tachometr periodicky každou vteřinu zasílá na konfigurovanou "cloud" metodu POST data, obsahující ujetou vzdálenost v centimetrech a aktuální průměrnou rychlost v kilometrech za hodinu, ve strojově čitelném formátu JSON. Příklad JSON dat spolu s HTTP hlavičkou je dostupný v soupisu \ref{code:json_data}.

\begin{listing}[h]
\begin{minted}[frame=single,
               framesep=3mm,
               linenos=true,
               xleftmargin=21pt,
               tabsize=4]{http}
POST / HTTP/1.1
Host: 172.20.10.13
User-Agent: ESP32HTTPClient
Connection: keep-alive
Content-Type: application/json
Content-Length: 52

{
"distanceCM": 20263.27,
"avgSpeedKmph": 22.54
}
\end{minted}
\caption{Příklad HTTP hlavičky s JSON daty} 
\label{code:json_data}
\end{listing}

% Highlight the advantages and benefits offered by the project.
\subsection{Omezení}\label{subsec:limits}
Aktuální implementace neumožňuje uživatelskou změnu obvodu kola, tato skutečnost však není vyžadována pro účel demonstrace funkčnosti řešení, v případě potřeby by bylo možné o tento údaj rozšířit formulář dostupný na administrativním webovém rozhraní.

Tachometr neumožňuje zobrazit ani nastavit aktuální čas. Časová značka se nevyskytuje v datech dostupných na webovém (aplikačním) rozhraní a opatření dat časovými značkami je ponecháno jako zodpovědnost návazným aplikacím pro sběr dat.

Tachometr si do perzistentní paměti neukládá časové průběhy dat dostupných na displeji nebo webovém (aplikačním) rozhraní, tachometr vystavuje vždy poslední aktuální data.

\end{document}